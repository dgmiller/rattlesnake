
\documentclass[12pt]{amsart}
\usepackage{geometry}
\usepackage{amsthm}
\geometry{a4paper}

\newtheorem{ex}{Example}
\newtheorem{quest}{Exercise}

\theoremstyle{definition}


\title{Scientific Programming in Python}
\author{An Introduction}
\date{}

\begin{document}

\maketitle
\tableofcontents

\section{Introduction}% intro, this is the prototype for the Spring 2015 class.

Throughout this course, the main objective is to learn to turn thought into code very quickly. Python is a good first language for this because of its very clear readability and strait-forward approach to programming. Due to a wide variety of differences in operating systems and other variables, there may be times that you need to go in search of help. Fortunately, Python has a vast community that you can use by searching for questions online. Finding a mentor will also help you sort through the information and figure out what to ask and how to solve problems. Much of the learning in this course is expected to be done via independent study; the tutorials and examples will contain all the basic information necessary to explore on your own. It is highly recommended that you seek out other materials to supplement your learning. Relying solely on the resources of this document will only give you a functional understanding of programming. To become fluent, you will need to do more than the requirements given to you here.

There are video tutorials that accompany each section. First, read the section contents and then watch the video tutorial. An example of solutions will be given in a Python file at the end of completing each section. These examples are not the only way the solution can be obtained; there may be more solutions that are more elegant and effective. The only wrong code, for our purposes, is the code that doesn't work. Use the examples to learn to read code and adopt techniques that you like.

Each section is followed by a list of exercises that provide an opportunity to use the tools learned in the section and to think in terms of code. It is recommended that you make your own document and take notes about the information you learn so you can refer back to it when needed. If you can learn to have fun and code everyday, you will master the concepts here which will build the base for successful future programming.

\section{The Command Line}% Lesson 1

The command line is how people used the computer before the advent of the GUI (Graphical User Interface).\footnote{A computer at a university, for example, would have the large mainframe computer housed in one building on campus. Giant cables would be connected from this machine to a "terminal" screen in a professor's office (if you were really important).} By entering a command we can directly communicate with the computer and tell it what we want it to do. This gives us more control over the computer allowing us to do things more efficiently than with a GUI.

Today, the command line is a program on the computer; we will use the Terminal (for Mac), PowerShell (for PC), and bash (for Linux). We will first use the command line to navigate through the computer's \textbf{directories} (what we usually call folders) and \textbf{files}. Naturally, this means we have to learn what commands to put into the command line. The following is a list of commands to implement on the command line:

\begin{itemize}
  \item pwd - "print working directory"; displays where you are in the computer (think of it as a "file path").
  \item ls - "list"; lists the directories and files of the directory you are working in.
  \item cd - "change directory"; moves you into a different specified directory. 
  \item mkdir - "make directory"; creates a new directory.
  \item mv - "move"; moves, copies or renames a file.
  \item rm - "remove"; permanently deletes a file or directory. Be careful with this one--you can't undo it!
  \item clear - wipes the screen of previous commands and output. It does not delete anything.
\end{itemize}

Note that if you have a directory with multiple words, such as "Python Code", you must refer to that directory in quotation marks.

\begin{quest}
Make a directory called \textbf{pycode} inside your Desktop directory.
\end{quest}
\begin{quest}
Make a simple text file called \textbf{firstone.txt} and type in the list of commands above for reference. Save it in your \textbf{pycode} directory.
\end{quest}

\section{Python Installation}% Lesson 2

Python is a high-level programming language. It is a very powerful and efficient language to use for scientific programming. What do we mean by all this? It partially has to do with the style and philosophy of a computer language called a \textbf{programming paradigm}.

Binary is the computer's native language. It deals with simple on or off syntax using 1 or 0 to tell the computer whether or not to turn on a particular electrical current. To avoid programming in this tedious language, computer scientists developed \textbf{imperative programming languages} such as FORTRAN which explicitly tell the computer to perform a sequence of steps in a specific way. This is still used today for high-performance computing where efficient use of energy is extremely important\footnote{For example, in weather forecasting on supercomputers.}.

Some programs, however, do not need to be carried out in a specific way. Other languages were developed to simplify the programming process. Some of the most widely used are \textbf{object-oriented programming languages} such as C++ and Java. These are still very efficient languages but allow programmers much more flexibility to perform certain tasks.

Still, these languages are not particularly easy to read or write. This is where Python comes in. Python supports all of the previously mentioned programming paradigms. It is much easier to read and, compared to most languages, Python requires significantly fewer lines of code to perform a task. Because of its versatility as a \textbf{multi-paradigm programming language}, Python is a kind of "swiss-army knife" language since it is able to perform a broad range of tasks in an elegant and efficient way. On the spectrum between binary and human languages, Python and other high-level programming languages are closer to human language than most other computer languages. This makes it easier to learn as well as a lot of fun.

\begin{quest}
Download and install Python on your computer from www.python.org.
\end{quest}
\begin{quest}
Open up a command line window and type \textbf{python} and hit ENTER. To exit, type \textbf{quit()} and hit ENTER.\footnote{If an error message appears, you may need to troubleshoot. There are lots of resources available online if you search for them in a search engine.}
\end{quest}

\section{Python Basics}

The command line is used by typing in commands for the computer to execute. Python is a language distinct from the command line. When we type in \textbf{python} into the command line, we leave the command line and are now dealing directly with the Python language. To go back to the command line, we must first quit Python mode by typing in \textbf{quit()} with the parentheses included.

In order to write large programs in Python code, we want to have some sort of program that we can type text into. This program is called a \textbf{text editor}. Every computer comes with a default text editor but there are many others available online. Below is a list of free text editors and Interactive Development Environments (IDE) that you can check out. For beginners, TextWrangler and Notepad++ are probably best:

\begin{itemize}
  \item TextWrangler
  \item Notepad++
  \item Vi or Vim
  \item Emacs
  \item Sublime
  \item Eclipse
  \item Spyder
  \item iPython Notebook
\end{itemize}

\begin{quest}
Watch the video tutorial.
\end{quest}
\begin{quest}
Create a file called \textbf{learn1.py} in your text editor and type in the code from the tutorial. Save it in your \textbf{pycode} directory, then run the file in Python. Name your Python files according to the \textbf{learn\#.py} format for convenience.
\end{quest}

\section{Data Types}%Data structures as subsection

Once you have your text editor downloaded and Python installed, it is now important to learn what to write. A \textbf{Data Type} is just what it implies--a type of data defined by certain properties or characteristics. When we write code, we turn these data types into an \textbf{Abstract Data Type (ADT)}. This allows the computer to perform certain operations with the data according to the properties inherent in the data type. The following are some examples of ADTs:

\begin{itemize}
  \item int - a whole number (integer).
  \item float - a general purpose decimal number.
  \item string - a list of characters contained within quotation marks.
  \item list - an organized sequence of data entries usually of a particular type.
  \item dict - similar to a list but ordered by attributes (like a dictionary)
\end{itemize}

\begin{quest}
Write a piece of code that spells your first name backwards.
\end{quest}
\begin{quest}
Write a piece of code that calculates the average of the numbers 512, 315, 720, 978 and 694.
\end{quest}

\section{Input and Output}% Lesson 4

\textbf{Input} is the information that we give to the computer while \textbf{Output} is the information that the computer gives us. This relates to the data types. If you give the computer input of the wrong data type, it will output an error message. It is important to distinguish between the different forms of input and output. For example, when you have stored a some value in a variable, you can ask the computer to output either the value that is stored in that variable or print the actual variable itself. This will be important when we start to learn about functions.

There are two ways to prompt the computer to ask you for input in Python. The first is \textbf{raw\_input()}. The computer will wait for you to enter something in and will store it as a string variable regardless of the data type you put in. You also don't have to put anything in and can simply press ENTER to move on. This is a great way to run only parts of your code at a time in a really big project to see if they are working correctly. The second way is \textbf{input()}. The computer will again wait for you to give it some input but will evaluate that input based on what you type in. With this way, it is necessary to be more specific if your input is a string\footnote{Avoid error messages with strings by putting your input in quotation marks} but numbers such as integers and floats are easily understood by \textbf{input()}. This means you don't have to change variables from strings to floats or integers but is more prone to error.

With both of these methods, you can give yourself a string prompt to remind you what to type. This will help you keep track of variables and make fewer errors. For example, the variable \textbf{a = raw\_input("Type your name: ")} when run by the computer will print out the message "Type your name: " and wait for you to enter your name.

\begin{quest}
Write a script that asks for a first and last name, computes the average number of characters in the name, and prints the average and their initials.
\end{quest}
\begin{quest}
Write a script that asks for someone's grades and computes their GPA.
\end{quest}

\section{Reading and Writing Files}% Lesson 5

One of the most valuable uses of code is to easily work with different kinds of files. For example, say we have the grades of a student and calculated their their GPA. Now we need to print their transcript in a format that the student can read without giving them just a copy of what the computer printed to the screen. We will primarily work with \textbf{.txt} and \textbf{.csv} files\footnote{Changing the dot notation ending on filenames does not change the structure of the file. It merely is a convention that helps the computer determine how to display the file. If the computer does not recognize the ending of the file, it will ask you to choose the program with which to open the file.}. The \textbf{.txt} file format is just a text file that stores characters; it is the most basic format for editing text. The \textbf{.csv} file format is known as a \textbf{Comma Separated Values} file. Each unique piece of information is separated by a comma in the stored file. This is easily read by a number of different programs such as Microsoft Excel which will open and display the \textbf{.csv} file data into a nice spreadsheet.

\begin{quest}
Write a script that prompts the user for input and writes it to a file called \textbf{readme.txt}. Then have your program read the script back to you.
\end{quest}
\begin{quest}
Write a script that takes in a student's grades and writes them to a \textbf{.csv} file. Try opening it in a spreadsheet program.
\end{quest}

\section{If and Else Statements}% Lesson 6

Suppose that our input might not be the same every time we enter the information. We need the computer to decide what to do with the data we give it at run time. This is where \textbf{if statements} come in handy. A good use for this would be to check for input errors. If the input is supposed to be a number but the user accidentally typed in a letter instead, the computer will think the input is a string and our code will break. Instead we could tell the computer: if the input is a number, then proceed with the computations but if it is a string, print an error message. There are other ways of doing this in Python by using the \textbf{try statement} which does a good job of looking for errors. We may also want the computer to do different things based on the input. We may want to store different data in different ways. The exercises and video tutorial will show you some examples of this.

\begin{quest}
Write a program that asks the user for input about 3 participants in a 5K. It doesn't matter whether the input is their name or their race bib number. Have the computer sort this out and display their 5K times based on the input. Write this information to a \textbf{.csv} file.
\end{quest}
\begin{quest}
Write a script that determines whether an input is a string or an integer or a float. If it is a string, print an error. If it is an integer, print the prime factorization of the integer. If it is a float, print the decimal remainder of the number.
\end{quest}

\section{Functions and Modules}% Lesson 7

It might feel pretty painful to write the same lines of code over and over and over and over again. This is why we use functions. You have seen them before but usually in the form of \textbf{quit()}, \textbf{input()} or \textbf{eval()}. Now we will build our own functions.

A function is always displayed with parentheses. These parentheses are where we will put our \textbf{parameters} which you can think of as the input of the function. Not all functions need parameters passed to them. The \textbf{raw\_input()} function works just fine without typing anything inside the parentheses. If we wanted to, we could type in a string to remind us what to type. The function then displays that information and asks for input. If we want the function to give us something back, then we will need to tell it to \textbf{return} a value or variable. The \textbf{return} statement will end the function so make sure everything happens inside the function before the \textbf{return} statement. Good coding practice is that if you are repeating a process three times in a piece of code, you should replace that process with a function.

A module is a kind of package or library that has some predefined functions that we can use. Commonly used modules are available for download and are included in the Python 2.7 download. Others can be downloaded for other specific purposes. Python has a vast community in the scientific programming field and have lots of different modules to choose from. In a bit, we will begin to work with the \textbf{numpy} and \textbf{scipy} modules. These will make our computations much easier to work with but for now, we will build our own modules (a script of predefined functions) to understand what they are and how they work.

\begin{quest}
Write a script of 3 functions that parses a sentence, sorts the words alphabetically and prints the list to the screen. In Python, import your script and makes string variables that are sentences and use these functions to sort the words.
\end{quest}
\begin{quest}
Write a script that uses a function to take a person's name and prints their initials. In your script, prompt the user for input and compute the initials of three different people.
\end{quest}

\section{Loops and Recursion}% Lesson 8

When we want the computer to repeat a certain task, we use loops. There are primarily two different kinds of loops. The first is a \textbf{for} loop. We use this loop when we want to do a something a finite number of times. When we want the program to keep going until a certain condition is met, we use a \textbf{while} loop. This is especially handy for making a menu of options with one of the options being 'quit'.

In the case of functions, sometimes we will do the same process over again but for different values. \textbf{Recursion} is a case in which a function calls itself. For example, say we want to know what $5!$ is. By definition, $5! = 5 * 4!$. In order to know this, though, we need to know $4! = 4 * 3!$ and so on. If our function computes factorials, then it would seem that it must call itself inside the function. We will not get muddy in the details of how this works on the computer for now. Understanding the basic ideas will help for further implementation down the road.

\begin{quest}
Write a script that asks the user for an integer number and returns the factorial of that number until the user types \textbf{quit}.
\end{quest}
\begin{quest}
Find the sum of all the multiples of $5$ and $7$ below $1000$. (Hint: use a \textbf{for} loop)

\section{Classes and Objects}% Lesson 9

\section{Inheritance}% Lesson 10





\end{document}